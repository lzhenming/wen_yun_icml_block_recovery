\section{Non-uniform block size case}

When $\ell_{\min} = \omega(d^{2/3}\xi^{4/3})$, there is a positive constant $c_0$ such that when $n \geq c_0 \ell_1$, we can detect the block using our algorithm in a recursive way.

Let $\ell_{i'}$ denote that node $i$ is in a block of size $\ell_{i'}$. Let $\ell_1(s)$ be the largest block size in the $s$-th iteration. We will first show that blocks larger than $\ell_1(s)/2$ can be detected with overwhelming probability, and then those blocks smaller than $\ell_1(s)/2$ will be reserved for the next iteration. Our analysis follows the same logic as the uniform block size case, and we only need to change the dimension of some random matrices to complete most proofs.

\subsection{Behavior of $\hat \Sigma_{i,k}$}
\begin{lemma}\label{lem:boundSigma_n} Let $i, k \in [d]$ and $\ell_{i'} \geq \ell_1(s)/2$. Let $\calf_i$ be the $\sigma$-algebra generated by $M_{i, :}$, $\mX$, and $E_{i,:}$.
we have 1. $\hat \Sigma_{i,k} | \calf_i$ follows a Gaussian distribution, namely $N(0, V^2_{i,k})$, and 2. the value $V_{i,k}$ depends on whether $i$ and $k$ are in the same block. Specifically, there exists a constant $c > 0$ such that
     
     \begin{itemize}
        \item If $i$ and $k$ are in the same block,  
            \begin{equation}
                n^2V^2_{i,k} \in_p (1\pm \frac{c\log d}{\ell_{i'}})\sigma^4\sigma^4_x[n\ell_{i'}(n + \ell_{i'} +1) \pm \xi^2 n^2 \ell_{i'}^{\frac{1}{2}}]. 
            \end{equation}
        \item If $i$ and $k$ are in different blocks, 
        \begin{equation}
             n^2V^2_{i,k} 
            \in_{p} (1 \pm \frac{c\log d}{\sqrt{\ell_{i'}\ell_{k'}}}) \sigma^4\sigma^4_x [n \ell_{i'}\ell_{k'} \pm \xi^2 n\ell_{i'}\ell_{k'}^{\frac{1}{2}}].
         \end{equation}
    \end{itemize}
\end{lemma}

\myparab{Proof of Lemma~\ref{lem:boundSigma_n}}
\begin{proof}
Note that two conditions are relaxed here. (i) $\ell_{i'}$ and $\ell_{k'}$ may not be the same; (ii) $\ell_{i'},\ell_{k'} = O(n)$ instead of $\ell_{i'},\ell_{k'} = \Theta(n)$.

\begin{itemize}
    \item \myparab{Leading term}
    When $i$ and $k$ are in the same block, the proof is identical to the uniform case and we only need to change $\ell$ into $\ell_{i'}$.
    When $i$ and $k$ are in different blocks, proof is almost the same except that $\mX_{(1)}$ is $\mX_{(i')} \in \reals^{n \times \ell_{i'}}$ and $B \in  \reals^{\ell_{i'}\times \ell_{k'}}$.By Eq.\ref{eqn:expect_n} we have the expectation as $n \ell_{i'}\ell_{k'}$.
    \item \myparab{Minor terms}
    For the minor terms, Lemma~\ref{prop:sameblock_term} still holds. But Lemma~\ref{prop:difblock_term} now is modified as follows because the dimension of some random matrices is changed.
  
\begin{corollary} \label{cor:difblock_term_n}
Let $Z \in \reals^{p_1}$ be a vector with entries following i.i.d. standard Gaussian distribution and $X_1 \in \reals^{n\times p_1}, X_2 \in \reals^{n\times p_2}$ be two independent matrices with entries as i.i.d. standard Gaussian random variables. Here $p_2 = O(p_1)$ and then let $c_2$ be a positive integer, we have

\begin{equation}
    Z(X_1^{\transpose}X_2X_2^{\transpose}X_1)^{2} Z^{\transpose}
    \in_p (\sqrt n \pm \sqrt p_1 \pm \xi)^{4}(p_1 p_2(1+p_1+p_2) \pm \xi^2 p_1^2 \sqrt{p_2})
\end{equation}

\begin{equation}
    Z(X_1^{\transpose}X_2)(X_2^{\transpose}X_2)^{c_2}(X_2^{\transpose}X_1) Z^{\transpose}
    \in_p (\sqrt n \pm \sqrt p_1 \pm \xi)^{2+2c_2}(p_1p_2 \pm \xi^2 p_1\sqrt{p_2})
\end{equation}
\end{corollary}
    Now with Lemma~\ref{prop:sameblock_term} and Corollary~\ref{cor:difblock_term_n}, we can analyze minor terms.
\end{itemize}
\end{proof}

\subsection{Behaviors of $\hat \Sigma_{i,k} \hat \Sigma_{j,k}$}

\myparab{Proof of Lemma~\ref{lem:interaction_n}}
\begin{proof}
We assume that $\ell_{i'},\ell_{j'} \in [\ell_{1}(s)/2, \ell_{1}(s)]$ and $\ell_{k'} = O(\ell_{1}(s))$. And we will follow the three steps in uniform block case.

\myparab{Step 1. Decomposition.}
\begin{corollary}\label{cor:alpha}
Using the above notation we have $|\alpha| = O\left(\log d/\sqrt{\ell_1(s)} \right)$.
\end{corollary}
We can finish our proof of Corollary~\ref{cor:alpha} using the same techniques as before with Lemma~\ref{lem:boundSigma_n}, Lemma~\ref{prop:sameblock_term}, Corollary~\ref{cor:differentterm2_n} and Corollary~\ref{cor:difblock_term_n}.

\begin{corollary}\label{cor:differentterm2_n}
Let $Z^{\transpose} \in \reals^{p_2}$ be a vector with entries following i.i.d. standard Gaussian distribution and $X_1 \in \reals^{n\times p_1}, X_2 \in \reals^{n\times p_2}, X_3 \in \reals^{n\times p_3}$ be three independent matrices with entries as i.i.d. standard Gaussian random variables. Then we have
\begin{equation}
\begin{split}
    &Z X_2^{\transpose}X_3X_3^{\transpose}X_1X_1^{\transpose}X_3X_3^{\transpose}X_2 Z^{\transpose}  \\ 
    \leq_p& (\sqrt{n} + \sqrt{p_1} + \xi)^{2}(\sqrt{n} + \sqrt{p_2} + \xi)^{2}(\sqrt{p_1}  +\sqrt{p_3} +  \xi)^{2}(\sqrt{p_2}  +\sqrt{p_3}+ \xi)^{2}(p_2+\xi\sqrt{p_2})
\end{split}
\end{equation}
\end{corollary}
This corollary is a modified version of  Lemma~\ref{prop: difblock_term2}.

\myparab{Step 2. Approximation.}
Corollary~\ref{cor:alpha} and $\ell_1(s) = \omega(d^{2/3}\xi^{4/3})$ imply that $|\alpha| = o(1)$. So Lemma~\ref{lem:calculate} holds here.

\myparab{Step 3. Application of first moment result.}
The error term in Lemma~\ref{lem:calculate}, $\frac{f_{K^{\bot}_i}(\tau )\alpha^2 \tau}{\sigma^2_{i,\bot}}$, is $ O(\alpha^2/\sigma^2_{i,\bot}) = O\left(\frac{n\log d}{\ell_1(s)\ell_{k'}} \right)$. 
\end{proof}






