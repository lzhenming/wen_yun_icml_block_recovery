
\subsection{Putting everything together}
{\color{red} MOVE TO APPENDIX.}

We now use Proposition~\ref{prop:interaction} to complete the proof of Theorem~\ref{thm:main}.  Specifically, we shall let {\color{blue} $\theta = \epsilon^2 d + \ell/2$} and show that \emph{(i)} When $i$ and $j$ are in the same block $|N(i) \cap N(j)| > \theta$ with overwhelming probability, and \emph{(ii)} When $i$ and $j$ are not in the same block, $|N(i) \cap N(j)| < \theta$ with overwhelming probability.


{\color{red} ZL's question: $\theta$ depends on $\epsilon$ but not $\epsilon_1$ and $\epsilon_2$ described In Sec 4.2? Also, the $\epsilon$ is the same as the noise term in our model so we need to consider changing the notation; maybe use $\mathbf{\varepsilon}$ as the noise?}
{\color{blue}by YW: sorry, I have not updated the part including $\theta$ depending on $\epsilon$, because there is something unsolved. And in my analysis, which uses $\epsilon_1$ and $\epsilon_2$, I didn't explicitly show $\theta$ but give the conditions under which there is a suitable one. If we cope with the unsolved problem, we can set $\theta$.}

Let $S_{i,j}(k)$ is an indicator variable that sets to 1 if and only if $\{i, k\}, \{j, k\} \in E(G)$. Also, recall that $\calf$ is the $\sigma$-algebra generated by $\mX, M_{i,:}, M_{j,:}, E_{i,:}, E_{j,:}$. Note that $\{S_{i,j}(k)\}_{k \neq i,j}$ are independent and bounded conditioned on $\calf$. We may apply a simple Chernoff bound to calculate $\sum_{k \neq i, j}S_{i,j}(k)\mid \calf$. 

Here, we compute the expectation of $\sum_{k \neq i, j}S_{i,j}(k)\mid \calf$ and show that there is a clear gap between the case when $i$ and $j$ are in the same block and that when $i$ and $j$ are in different blocks. {\color{red} In Appendix X,} we shall analyze the concentration property. Specifically, we shall show the tails of $\sum_{k \neq i, j}S_{i,j}(k)\mid \calf$ are substantially smaller than the gap; therefore, with overwhelming probability we can determine whether $i$ and $j$ are in the same block

\myparab{Expectation.} We first analyze the case, where \emph{$i$ and $j$ are in the same block}.  Wlog, assume $i,j \in B_1$. Let $P_{11} \equiv \E[S_{i,j}(k)\mid \calf,i,j,k\in B_1]$ and $P_{12} \equiv \E[S_{i,j}(k)\mid \calf, i,j\in B_1,k \in B_2]$. By Proposition~\ref{prop:interaction}, we have 
\begin{equation}
    P_{11} \geq_p 0.9(1-\epsilon_1)^2 - \frac{c_3 n \log^2d}{ \ell^{3}} \quad \mbox{ and } \quad 
    P_{12} \geq_p 0.9\epsilon^2_2 - \frac{c_3 n \log^2d}{ \ell^{3}}
\end{equation}

We next analyze the case, where \emph{$i$ and $j$ are not in the same block}.  Wlog, assume $i \in B_1 $ and $j \in B_2$. Let $P_{21} \equiv \E[S_{i,j}(k)\mid \calf,i,k\in B_1,j\in B_2]$ and $P_{22} \equiv \E[S_{i,j}(k)\mid \calf, i\in B_1,j\in B_2,k \in B_3]$. By Proposition~\ref{prop:interaction} again, we have 
\begin{equation}
    P_{21} \leq_p 1.1(1-\epsilon_1)\epsilon_2 \quad \mbox{ and } \quad 
    P_{22} \leq_p 1.1 \epsilon^2_2. 
\end{equation}

One can check that the gap $P_{11} + P_{12} - P_{21} - P_{22}$ is sufficiently large. Together with the tail analysis presented in {\color{red} Appendix X}, we complete our proof for Theorem~\ref{thm:main}.

{\color{blue} the above indices are buggy. We should write $k \in B_1, k\neq i,j$; Also, it is really $(1-\epsilon)^2 (\ell-2)$. We will need to find a way to make the notation nice. }

